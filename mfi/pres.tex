\documentclass[t]{beamer}
%\usepackage{multimedia}		% for movies, sounds, animations...
%\usepackage{babel}		% new german
\usepackage[utf8]{inputenc}		% input...


\usepackage[uniWZ, blockBG, tams]{tamsBeamer}
%-----------------------------------------------------------------------------
%-- options		------------------------------------------------------
%			tams	|	- TAMS		publication
%			cinacs		- CINACS	publication
%			engl		- english strings	[german]
%			uniWZ	|	- uni		watermark
%			tamsWZ	|	- tams+uni	watermark
%			cinacsWZ	- cinacs+uni	watermark
%			secToc	|	- toc repetition at each section
%			secTocA		- -"-, all sections: show
%					  replacement for toc in short docs
%			subsecToc	- toc repetition at each subsection
%			secNum  	- (sub)-section numbering
%			fullstep	- always step through items
%			noFoot		- footline	off
%			noPage		- page numbers	off
%			noAuth		- author	off
%			conference	- footline with \foottitle{...}
%			blockBG	|	- block, example etc. background
%			blockRound	- -"-, rounded+shadow


%%%%%%%%%%%%%%%%%%%%%%%%%%%%%%%%%%%%%%%%%%%%%%%%%%%%%%%%%%%%%
% Packages:
\usepackage{graphicx}                   % Inclusion of graphics
\DeclareGraphicsExtensions{.pdf,.eps,.png,.jpg} 
\graphicspath{{./fig/}}                 % Path to a folder where all pictures are located

\usepackage{amsfonts}                   % AMS Math Packet (Fonts)
\usepackage{amsmath}                    % AMS Math Packet
\usepackage{amssymb}                    % Additional mathematical symbols
\usepackage{amsthm}

\DeclareMathOperator*{\argmin}{arg\,min}
\DeclareMathOperator*{\argmax}{arg\,max}

%%%%%%%%%%%%%%%%%%%%%%%%%%%%%%%%%%%%%%%%%%%%%%%%%%%%%%%%%%%%%
\usepackage{tikz}
\usepackage{tikzscale}
\usepackage{gnuplot-lua-tikz}

\newcommand{\includetexfig}[1]{\input{fig/#1.tex}}

\usetikzlibrary{shapes,fit,positioning}

\tikzstyle{state}=[circle,draw,thick]
\tikzstyle{observation}=[regular polygon,regular polygon sides=4,inner sep=1pt,draw,thick]

\tikzstyle{edata}=[draw,thick,rounded corners,minimum height=.7cm,text width=1.5cm,align=center]
\tikzstyle{data}=[above,text width=1.5cm,align=left]
\tikzstyle{process}=[rectangle,draw,thick,minimum height=.7cm,text width=1.5cm,align=center]
\tikzstyle{io}=[]

%%%%%%%%%%%%%%%%%%%%%%%%%%%%%%%%%%%%%%%%%%%%%%%%%%%%%%%%%%%%%

% fonts definitions			--------------------------------------
% ----------------------------------------------------------------------------
% default: cmss, OT1 fontenc		good with UniHH font "The Sans"
%					-> don't change fonts!
%\usepackage{times}			% other fonts
%\usepackage[T1]{fontenc}		%

% document definitions			--------------------------------------
% ----------------------------------------------------------------------------
\title{Multimodal Object Recognition from Visual and Audio Sequences}
%%{Beispiele zum Layout}

\author{Weipeng He, Haojun Guan and Jianwei Zhang}

%\author[AutorA, AutorB]		% author	-- option: short
% {A.~Autor\inst{1} \and B.~Autor\inst{2}}%		-- option: \inst{...}
% style option: [tams] predefines institute...
% or define \institute{...}
%					% \inst{...} for different institutions
%\institute[Universities A and B]	% institution	-- option: short
%{ \inst{1}%
%  University of A\\
%  Department of A
%  \and
%  \inst{2}%
%  University of B\\
%  Department of B}

\date{September 15, 2015}

\subject{TAMS, LaTeX, Folien}		% subject	-- option for pdf


% document starts here			--------------------------------------
% ----------------------------------------------------------------------------
\begin{document}

% titlepage				--------------------------------------
%\frame[plain]{\titlepage}		% suppress head- and footlines
\frame{\titlepage}

\section{Introduction}
\begin{frame}
  \frametitle{Motivation}

  \begin{itemize}
    \item Human use multiple sensory information to recognize objects.

      ~
    \item Difficulties of visual object recognition:
      \begin{itemize}
        \item Sensitive to image transformations. 
      \end{itemize}

      ~
    \item Sound provides complementary information.
      \begin{itemize}
        \item Make sound via interactions.
        \item Information about material. 
        \item Category by function: objects that make sound.
      \end{itemize}
  \end{itemize}
\end{frame}

\iffalse
\begin{frame}
  \frametitle{Related Work}

  \begin{itemize}
    \item Sinapov and his colleagues~\cite{sinapov_interactive_2009,sinapov_object_2011}: 
      \begin{itemize}
        \item Audio and proprioceptive information.
        \item Relational learning.

          ~
        \item No visual information.
      \end{itemize}

      ~
    \item Nakamura and his colleagues~\cite{nakamura_multimodal_2007,nakamura_bag_2012}:
      \begin{itemize}
        \item Visual, audio and haptic information.
        \item Multimodal pLSA.

          ~
        \item Unsupervised categorization.
      \end{itemize}
  \end{itemize}
\end{frame}
\fi

\begin{frame}
  \frametitle{Objective}

  \begin{itemize}
    \item Build an object recognition system based on visual and audio information.

      ~
      \begin{itemize}
        \item What kind of features?

          ~
        \item How to make classification?

          ~
        \item How to combine multimodal information?
      \end{itemize}
  \end{itemize}
\end{frame}

\iffalse
\begin{frame}
  \frametitle{Two Tasks of Object Recognition}

  \begin{itemize}
    \item Specific object recognition.
      \includegraphics[width=\linewidth]{specific.tikz}

    \item Generic category recognition.
      \includegraphics[width=\linewidth]{generic.tikz}
  \end{itemize}
\end{frame}
\fi

\section{Feature Extraction}
\iffalse
\begin{frame}
  \frametitle{Bag-of-Words Model with SIFT Descriptors}

  \begin{itemize}
    \item SIFT Descriptor~\cite{lowe_object_1999}. 
      \begin{itemize}
        \item Local feature descriptor invariant to several image transforms.
        \item Standard for object recognition in robotics.

          ~
        \item Does not describe the whole image/object.
        \item Does not encode general features of a category.
      \end{itemize}
  \end{itemize}

  \centering
  \includegraphics[width=.7\linewidth]{mug2m}
\end{frame}
\fi

\begin{frame}
  \frametitle{Bag-of-Words Model with SIFT Descriptors}

  \begin{columns}
    \begin{column}{.46\linewidth}
      \begin{itemize}
        \item Bag-of-words model. 
        \item An image : a collection of local features.
      \end{itemize}
      \centering
      \includegraphics[width=.5\columnwidth]{mug2k}
    \end{column}

    \begin{column}{.46\linewidth}
      \begin{itemize}
        \item Vector quantization with a codebook.
          \begin{itemize}
            \item
              \includegraphics[width=.1\columnwidth]{mug2v11}~ : 1
            \item
              \includegraphics[width=.1\columnwidth]{mug2v21}~
              \includegraphics[width=.1\columnwidth]{mug2v22}~
              \includegraphics[width=.1\columnwidth]{mug2v23}~ : 3
            \item
              \includegraphics[width=.1\columnwidth]{mug2v31}~
              \includegraphics[width=.1\columnwidth]{mug2v32}~
              \includegraphics[width=.1\columnwidth]{mug2v33}~ : 3
            \item
              \includegraphics[width=.1\columnwidth]{mug2v41}~
              \includegraphics[width=.1\columnwidth]{mug2v42}~ : 2
            \item \ldots
          \end{itemize}
      \end{itemize}
    \end{column}
  \end{columns}
\end{frame}

\begin{frame}
  \frametitle{Mel-frequency Cepstrum Coefficients}

  \begin{itemize}
    \item DCT of log power spectrum at mel-scale.

      ~
    \item MFCCs are common acoustic features for audio processing, such as speech recognition and speaker recognition.
  \end{itemize}
\end{frame}

\section{Hidden Markov Models}
\begin{frame}
  \frametitle{Hidden Markov Models}

  \begin{itemize}
    \item During interactions images and sound change over time.
    \item HMM describes a distribution of a stochastic process.
      \begin{itemize}
        \item Applications in many fields including speech recognition, gesture recognition and so on. 
      \end{itemize}
      ~

    \item HMM is a generative model.
      \begin{itemize}
        \item Allows further inference for modality fusion. 
      \end{itemize}
  \end{itemize}
\end{frame}

\begin{frame}
  \frametitle{Classification with HMM}

  \begin{itemize}
    \item Estimate $\lambda^c$ with all data of class $c$.

    \item Approximate likelihood with learned model:
      \[ P(\mathbf{x}|c) = P(\mathbf{x}|\lambda^c) . \]
    \item Specific object recognition (multiclass classification):
      \[ f(\mathbf{x}) = \argmax_{c} P(\mathbf{x}|c) \]
    \item Generic category recognition (binary classification):
      \[ 
        f(\mathbf{x}) = \left\{
          \begin{array}{l l}
            +1, & \quad P(c=+1|\mathbf{x}) > \theta; \\
            -1, & \quad \text{otherwise.}
          \end{array} \right.
      \]
        where
      \[ P(c|\mathbf{x}) = \frac{P(\mathbf{x}|c)P(c)}{\sum_{c \in \{-1,+1\}} P(\mathbf{x}|c)P(c)} . \]
  \end{itemize}
\end{frame}

\section{Bimodal Object Recognition}
\begin{frame}
  \frametitle{Multimodal Fusion Methods}

  \begin{itemize}
    \item $\mathbf{x} = (\mathbf{v},\mathbf{a})$.
    \item The goal is to compute the joint likelihood:
      \[ P(\mathbf{v},\mathbf{a}|c) \]

      ~
    \item Two approaches:
      \begin{itemize}
        \item Feature Fusion.
        \item Decision Fusion.
      \end{itemize}
  \end{itemize}
\end{frame}

\begin{frame}
  \frametitle{Feature Fusion}

  \begin{center}
    \tiny
    \includegraphics[width=.7\linewidth]{featurefs.tikz}
  \end{center}

  \begin{itemize}
    \item Directly learn the joint likelihood with concatenated features:
      \[ P(\mathbf{v},\mathbf{a}|c) = P(\mathbf{v},\mathbf{a}|\lambda_{va}^c) \]
  \end{itemize}
\end{frame}

\begin{frame}
  \frametitle{Decision Fusion}

  \begin{center}
    \tiny
    \includegraphics[width=.7\linewidth]{decisionfs.tikz}
  \end{center}

  \begin{itemize}
    \item Learn separate models and combine them under conditional independence assumption:
      \[ P(\mathbf{v},\mathbf{a}|c) = P(\mathbf{v}|c) P(\mathbf{a}|c) = P(\mathbf{v}|\lambda_v^c) P(\mathbf{a}|\lambda_a^c) \]
  \end{itemize}
\end{frame}

\begin{frame}
  \frametitle{Comparison of Fusion Methods}

  \centering
  \begin{tabular}{p{.42\linewidth}|p{.42\linewidth}}
    \hline
    \multicolumn{1}{c|}{\bfseries Feature Fusion} & \multicolumn{1}{c}{\bfseries Decision Fusion} \\ \hline
    Theoretically correct. & Independent assumption needs to be hold. \\
    \\
    Models are complex. & Models are simple. \\
    \\
    Synchronization is \mbox{necessary}. & Synchronization is not necessary. \\ \hline
  \end{tabular}

\end{frame}

\section{Experiment}
\begin{frame}
  \frametitle{Experiment Setup}

  \begin{columns}
    \begin{column}{.4\linewidth}
      \begin{itemize}
        \item 33 household objects.

          ~

        \item Interactions:
          \begin{itemize}
            \item Strike with a stick.
            \item Push.
            \item Shake.
          \end{itemize}
      \end{itemize}
    \end{column}
    \begin{column}{.55\linewidth}
      \begin{tabular}[t]{lc}
        \hline
        Mugs & \includegraphics[width=.14\linewidth]{object2} ~ \includegraphics[width=.14\linewidth]{object3} ~ \includegraphics[width=.14\linewidth]{object6} \\
        Bottles & \includegraphics[width=.14\linewidth]{object16} ~ \includegraphics[width=.14\linewidth]{object19} ~ \includegraphics[width=.14\linewidth]{object20} \\
        Plastic & \includegraphics[width=.14\linewidth]{object16} ~ \includegraphics[width=.14\linewidth]{object17} ~ \includegraphics[width=.14\linewidth]{object33} \\
        Metal & \includegraphics[width=.14\linewidth]{object9} ~ \includegraphics[width=.14\linewidth]{object10} ~ \includegraphics[width=.14\linewidth]{object12} \\
        Fragile & \includegraphics[width=.14\linewidth]{object1} ~ \includegraphics[width=.14\linewidth]{object2} ~ \includegraphics[width=.14\linewidth]{object7} \\
        Containers & \includegraphics[width=.14\linewidth]{object19} ~ \includegraphics[width=.14\linewidth]{object24} ~ \includegraphics[width=.14\linewidth]{object29} \\
        \hline
      \end{tabular}
    \end{column}
  \end{columns}
\end{frame}

\begin{frame}
  \frametitle{Specific Object Recognition Result}

  \centering
  \begin{tabular}[h!]{c|c}
    \hline
    Method & Accuracy \\ \hline \hline
    Feature Fusion & \textbf{95.8\%} \\
    Decision Fusion  & 95.7\% \\
    Visual Only & 86.7\% \\
    Audio Only & 83.6\% \\
    \hline
  \end{tabular}

  ~
  \begin{itemize}
    \item 5-fold cross validation.
    \item HMM with 2 states, 6 mixture components and diagonal covariance matrix.
  \end{itemize}
  \vfill
  {\scriptsize
    \[ \text{accuracy} =  \frac{\text{\# correctly classified}}{\text{\# total}} \]
  }
\end{frame}

\iffalse
\begin{frame}
  \frametitle{Generic Category Recognition Result}
  \centering
  \begin{tabular}[h]{l*{5}{p{.07\linewidth}}p{.11\linewidth}}
    \toprule
    & \multicolumn{6}{c}{Categories} \tabularnewline \cmidrule(r){2-7}
    \multicolumn{1}{c}{Method} & Mugs & Bottles & Plastic & Metal & Fragile & Containers \tabularnewline \midrule
    Feature Fusion & 0.750 & \textbf{0.802} & 0.820 & 0.830 & 0.711 & 0.671 \tabularnewline
    Decision Fusion & \textbf{0.771} & 0.798 & \textbf{0.839} & 0.870 & 0.697 & \textbf{0.675} \tabularnewline
    Visual Only & 0.763 & 0.778 & 0.813 & \textbf{0.877} & 0.590 & 0.567 \tabularnewline
    Audio Only & 0.642 & 0.707 & 0.732 & 0.772 & \textbf{0.819} & 0.620 \tabularnewline
    \bottomrule
  \end{tabular}
  \begin{itemize}
    \item Area under the curve (AUC) of the receiver operating characteristic (ROC) curve.
  \end{itemize}
\end{frame}

\begin{frame}
  \frametitle{Generic Category Recognition Result}

  \begin{columns}
    \begin{column}{.5\linewidth}
      \centering
      \scriptsize
      \includegraphics[width=\columnwidth]{roc.tikz}
    \end{column}
    \begin{column}{.5\linewidth}
      \begin{itemize}
        \item Object-based cross validation.
        \item Receiver operating characteristic (ROC).
        \item Area under the curve (AUC).
          \begin{itemize}
            \item 0.5 => Random
            \item 1.0 => Perfect 
          \end{itemize}
      \end{itemize}
    \end{column}
  \end{columns}
  {\scriptsize
    \[ \text{true positive rate} =  \frac{\text{\# true positive}}{\text{\# conditional positive}} \]
    \[ \text{false positive rate} =  \frac{\text{\# false positive}}{\text{\# conditional negative}} \]
  }
\end{frame}
\fi

\begin{frame}
  \frametitle{Generic Category Recognition Result}

  \begin{columns}
    \begin{column}{.6\linewidth}
      \centering
      \footnotesize
      \includegraphics[width=\columnwidth]{mug.tikz}
    \end{column}

    \begin{column}{.4\linewidth}
      \vspace{-1.4\linewidth}
      \begin{itemize}
        \item Mugs
      \end{itemize}
      ~

      \begin{tabular}[h]{c|c}
        \hline
        Method & AUC \\ \hline \hline
        Feature Fusion & 0.750 \\ \hline
        Decision Fusion  & \textbf{0.771} \\ \hline
        Visual Only & 0.763 \\ \hline
        Audio Only & 0.642 \\ \hline
      \end{tabular}
    \end{column}
  \end{columns}
\end{frame}

\begin{frame}
  \frametitle{Generic Category Recognition Result}

  \begin{columns}
    \begin{column}{.6\linewidth}
      \centering
      \footnotesize
      \includegraphics[width=\columnwidth]{bottle.tikz}
    \end{column}
    \begin{column}{.4\linewidth}
      \vspace{-1.4\linewidth}
      \begin{itemize}
        \item Bottles
      \end{itemize}
      ~

      \begin{tabular}[h]{c|c}
        \hline
        Method & AUC \\ \hline \hline
        Feature Fusion & \textbf{0.802} \\ \hline
        Decision Fusion  & 0.798 \\ \hline
        Visual Only & 0.778 \\ \hline
        Audio Only & 0.707 \\ \hline
      \end{tabular}
    \end{column}
  \end{columns}
\end{frame}

\begin{frame}
  \frametitle{Generic Category Recognition Result}

  \begin{columns}
    \begin{column}{.6\linewidth}
      \centering
      \footnotesize
      \includegraphics[width=\columnwidth]{plastic.tikz}
    \end{column}
    \begin{column}{.4\linewidth}
      \vspace{-1.4\linewidth}
      \begin{itemize}
        \item Plastic objects
      \end{itemize}
      ~

      \begin{tabular}[h]{c|c}
        \hline
        Method & AUC \\ \hline \hline
        Feature Fusion & 0.820 \\ \hline
        Decision Fusion & \textbf{0.839} \\ \hline
        Visual Only & 0.813 \\ \hline
        Audio Only & 0.732 \\ \hline
      \end{tabular}
    \end{column}
  \end{columns}
\end{frame}

\begin{frame}
  \frametitle{Generic Category Recognition Result}

  \begin{columns}
    \begin{column}{.6\linewidth}
      \centering
      \footnotesize
      \includegraphics[width=\columnwidth]{metal.tikz}
    \end{column}
    \begin{column}{.4\linewidth}
      \vspace{-1.4\linewidth}
      \begin{itemize}
        \item Metal objects
      \end{itemize}
      ~

      \begin{tabular}[h]{c|c}
        \hline
        Method & AUC \\ \hline \hline
        Feature Fusion & 0.830 \\ \hline
        Decision Fusion  & 0.870 \\ \hline
        Visual Only & \textbf{0.877} \\ \hline
        Audio Only & 0.772 \\ \hline
      \end{tabular}
    \end{column}
  \end{columns}
\end{frame}

\begin{frame}
  \frametitle{Generic Category Recognition Result}

  \begin{columns}
    \begin{column}{.6\linewidth}
      \centering
      \footnotesize
      \includegraphics[width=\columnwidth]{fragile.tikz}
    \end{column}
    \begin{column}{.4\linewidth}
      \vspace{-1.4\linewidth}
      \begin{itemize}
        \item Fragile objects (glass and ceramic)
      \end{itemize}
      ~

      \begin{tabular}[h]{c|c}
        \hline
        Method & AUC \\ \hline \hline
        Feature Fusion & 0.711 \\ \hline
        Decision Fusion  & 0.697 \\ \hline
        Visual Only & 0.590 \\ \hline
        Audio Only & \textbf{0.819} \\ \hline
      \end{tabular}
    \end{column}
  \end{columns}
\end{frame}

\begin{frame}
  \frametitle{Generic Category Recognition Result}

  \begin{columns}
    \begin{column}{.6\linewidth}
      \centering
      \footnotesize
      \includegraphics[width=\columnwidth]{nonempty.tikz}
    \end{column}

    \begin{column}{.4\linewidth}
      \vspace{-1.4\linewidth}
      \begin{itemize}
        \item Containers with contents
      \end{itemize}
      ~

      \begin{tabular}[h]{c|c}
        \hline
        Method & AUC \\ \hline \hline
        Feature Fusion & 0.671 \\ \hline
        Decision Fusion  & \textbf{0.675} \\ \hline
        Visual Only & 0.567 \\ \hline
        Audio Only & 0.620 \\ \hline
      \end{tabular}
    \end{column}
  \end{columns}
\end{frame}

\section{Conclusion}
\begin{frame}
  \frametitle{Conclusion}

  \begin{itemize}
    \item For specific object recognition, both fusion methods increased the accuracy for about 10\%, comparing to unimodal methods.

      ~
    \item For generic category recognition, both fusion methods increased the performance under the condition of neither of the modality is noise.

      ~
    \item There is no significant difference in performance between the two fusion methods.
  \end{itemize}
\end{frame}

\begin{frame}
  \frametitle{Future Work}

  \begin{itemize}
    \item Real-time recognition.
      \begin{itemize}
        \item Using SURF instead of SIFT.
      \end{itemize}

      ~
    \item Increase dataset.
      \begin{itemize}
        \item Collect data by robots or from Internet sources.
        \item More data make it possible to use discriminative methods to learn combination rules.
      \end{itemize}

      ~
    \item Learn good features using multimodal signals.
      \begin{itemize}
        \item Congruency as a heuristic.
      \end{itemize}
  \end{itemize}
\end{frame}

\end{document}

