\chapter{Background}\label{background}

In this chapter, three related fields regarding visual-audio object recognition will be reviewed. First, the traditional methods of object recognition based on visual information exclusively will be introduced. Then, the state-of-the-art in multimodal object recognition will be presented. Finally, the application of visual-audio integration in speech recognition will be explained.

\section{Visual Object Recognition}
Visual object recognition has been extensively studied in the recent decade. Vision is very essential for both humans and artificial agents to recognize an object, since it provides the most direct and informational clue about the object. Numerous methods have been proposed for visual object recognition and they have been used in a variety of applications.

Concerning object recognition, there are two types of task been studied: specific case and generic object category case \cite{grauman_visual_2011}. The specific object recognition is the task of recognizing a particular object, place or person from a giving image. For instance, recognizing your neighbor's car, the Eiffel tower or a ten euro note. These objects have little variation in appearance and shape. Unlike the specific case, generic object category recognition cope with classifying an instance to a category, for example to tell an instance is a car or a coffee mug. In this case, the objects belongs to the same category have the same conceptual meaning, but may vary in appearance and shape. Different methods are used to deal with these two types of task. In general, they use different features and algorithms. Now, we will discuss some of the methods.

One of the approaches which have achieved robust and efficient result in specific object recognition is proposed by Lowe \cite{lowe_object_1999}. His approach first extracts SIFT features from the image. Scale Invariant Feature Transform (SIFT)is a kind of local feature that is invariant to image translation, scaling, and rotation. Then, the extracted features are matched between train and test images. In the final step, the matched features are verified to confirm they are in a consistent geometric configuration. This approach has been widely used in robotic visual systems \cite{grauman_visual_2011}.

While specific object recognition can be solved as matching of local features, the task of object category recognition requires representations that capture the common characteristics of different instances in the same class. Example systems for object category recognition include the Viola-Jones face detector \cite{viola_rapid_2001}, bag-of-words model \cite{csurka_visual_2004} and HOG person detector \cite{dalal_histograms_2005}. 

The bag-of-words model \cite{csurka_visual_2004} for object category recognition is motivated by the bag-of-words model in text categorization. The visual image with a collection of local feature descriptors is analog to a document with a collection of words. In practice, a vocabulary of visual words is first constructed with clustering the feature descriptors from a number of images. Here, descriptors like SIFT or SURF can be used. Then, an image can be represented as a histogram of the vocabulary. After that, classifiers like support vector machine or Na\"ive Bayes can be used for categorization.

\section{Multimodal Object Recognition}

\section{Visual-Audio Speech Recognition}

