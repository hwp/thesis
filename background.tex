\chapter{Background}\label{background}

In this chapter, three related fields regarding visual-audio object recognition will be reviewed. First, the traditional methods of object recognition based on visual information exclusively will be introduced. Then, the state-of-the-art in multimodal object recognition will be presented. Finally, the application of visual-audio integration in speech recognition will be explained.

\section{Visual Object Recognition}
Visual object recognition has been extensively studied in the recent decade. Vision is very essential for both humans and artificial agents to recognize an object, since it provides the most direct and informational clue about the object. Numerous methods have been proposed for visual object recognition and they have been used in a variety of applications.

Concerning object recognition, there are two types of task been studied: specific case and generic object category case \cite{grauman_visual_2011}. The specific object recognition is the task of recognizing a particular object, place or person from a giving image. For instance, recognizing your neighbor's car, the Eiffel tower or a ten euro note. These objects have little variation in appearance and shape. Unlike the specific case, generic object category recognition cope with classifying an instance to a category, for example to tell an instance is a car or a coffee mug. In this case, the objects belongs to the same category have the same conceptual meaning, but may vary in appearance and shape. Different methods are used to deal with these two types of task. In general, they use different features and algorithms. Now, we will discuss some of the methods.

\subsection{Visual Specific Object Recognition}
One of the approaches which have achieved robust and efficient result in specific object recognition is proposed by Lowe \cite{lowe_object_1999}. His approach first extracts SIFT features from the image. Scale Invariant Feature Transform (SIFT)is a kind of local feature that is invariant to image translation, scaling, and rotation. Then, the extracted features are matched between train and test images. In the final step, the matched features are verified to confirm they are in a consistent geometric configuration. This approach has been widely used in robotic visual systems \cite{grauman_visual_2011}.

\subsection{Visual Object Category Recognition}
While specific object recognition can be solved as matching of local features, the task of object category recognition requires representations that capture the common characteristics of different instances in the same class. Example systems for object category recognition include the Viola-Jones face detector \cite{viola_rapid_2001}, bag-of-words model \cite{csurka_visual_2004} and HOG person detector \cite{dalal_histograms_2005}. 

The bag-of-words model \cite{csurka_visual_2004} for object category recognition is motivated by the bag-of-words model in text categorization. The visual image with a collection of local feature descriptors is analog to a document with a collection of words. In practice, a vocabulary of visual words is first constructed with clustering the feature descriptors from a number of images. Here, descriptors like SIFT or SURF can be used. Then, an image can be represented as a histogram of the vocabulary. After that, classifiers like support vector machine or Na\"ive Bayes can be used for categorization. This bag-of-words model achieved high and robust categorization accuracy.

Sivic et al. \cite{sivic_discovering_2005} used the bag-of-words model for unsupervised object categorization. Same as \cite{csurka_visual_2004}, given a collection of images, the feature of the images is first extracted as a bag of visual words. Then, generative models like Latent Semantic Analysis (pLSA) and Latent Dirichlet Allocation (LDA) are learned from the representation vectors. These models, which are successfully used for text analysis finding document topics, can also be used for finding the underlying common features of an object category. Since their learning method is unsupervised, it is applicable to a large unlabeled image set.

\section{Multimodal Object Recognition}
When thinking of how humans recognize objects, we use not only visual information, but also perceptions of other modalities, like auditory, haptic and olfactory perception. These extra perceptions provide complementary information. For example, a paper cup might have the same shape as a ceramic cup, and it is difficult to distinguish them only with vision. However, if we make some sound of them by knocking or touch them, we can use auditory or haptic information to tell the difference. This idea of multimodal object recognition is also applied to robotic systems.

\subsection{Multimodal pLSA Model}
Nakamura et al. \cite{nakamura_multimodal_2007} proposed an unsupervised object categorization method for robots based on visual, audio and haptic information. They used a multimodal pLSA model, which is extended from the pLSA model described in \cite{sivic_discovering_2005}, for the categorization.

In their experiment, a robot hand was used to grasp and shake the objects. During the interaction, visual, audio and haptic information were collected by a camera at the side, a microphone and pressure sensors on the robot finger. Afterwards, the data were processes separately as follows:
\begin{description}
  \item[Visual Information] \hfill \\
    Each image frame was represented using a bag-of-words model with scale and affine invariant salient region detector \cite{mikolajczyk_scale_2004} and SIFT descriptor \cite{lowe_distinctive_2004}. The visual vocabulary was generated by clustering the local descriptors in a collection of images into 600 clusters using k-means algorithm. The collection of images are 100 indoor images, which are independent of the objects in their experiment.
  \item[Audio Information] \hfill \\
    For each audio frame, a 13-dimensional MFCC feature was first calculated and then represented as a bag-of-words model with a vocabulary of 50 clusters. The vocabulary was generated with speech and noises.
  \item[Haptic Information] \hfill \\
    For each grasp of an object, a 2-dimensional feature vector was collected by the pressure sensors on the robot finger. Then, the feature vectors were vector quantized with 5 clusters, which was generated in advance.
\end{description}

They tested the unsupervised categorization on 40 objects in 8 categories. The objects were toys like stuffed toys, rattles, tambourines and balls. It was shown that their categorization result of using three modalities was identical to the categorization result of human subject. However, categorization using one modality alone did not work as well.

Furthermore, they also tested the performance of category and property inference of an unseen object. By using ``fold in'' heuristic \cite{hofmann_probabilistic_1999}, they correctly inferred the categories of all the objects in leave-one-out cross validation. For the property inference, their method can infer the auditory and haptic properties when given only the visual cue of the object.

\section{Visual-Audio Speech Recognition}

