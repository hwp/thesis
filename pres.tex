\documentclass{beamer}

\usepackage{pres}

\title{\trtitle \\ \vspace{5pt} \footnotesize \trtype}
\author{\trauthor \\ \vspace{5pt} \footnotesize \trfach}
\date{28 April, 2015}

\begin{document}

\frame{\titlepage}

\frame{
  \frametitle{Outline}
  \tableofcontents
}

\section{Introduction}
\begin{frame}
  \frametitle{Motivation}

  \begin{itemize}
    \item Difficulties of visual object recognition:
      \begin{itemize}
        \item Sensitive to image transformations. 
        \item Transparent objects.
      \end{itemize}
      ~

    \item Sound provides complementary information.
      \begin{itemize}
        \item Make sound via interactions.
        \item Information about material. 
        \item Category by function: objects that make sound.
      \end{itemize}
  \end{itemize}
\end{frame}

\begin{frame}
  \frametitle{Related Work}

  \begin{itemize}
    \item Sinapov and his colleagues~\cite{sinapov_interactive_2009,sinapov_object_2011}: 
      \begin{itemize}
        \item Audio and haptic information.
        \item Relational learning.

          ~
        \item No visual information.
      \end{itemize}
      ~

    \item Nakamura and his colleagues~\cite{nakamura_multimodal_2007,nakamura_bag_2012}:
      \begin{itemize}
        \item Visual, audio and haptic information.
        \item Multimodal pLSA.

          ~
        \item Unsupervised categorization.
      \end{itemize}
  \end{itemize}
\end{frame}

\begin{frame}
  \frametitle{Objective}

  \begin{itemize}
    \item Build a object recognition system based on visual and audio information.
      \begin{itemize}
        \item What kind of features?
        \item How to make classification?
        \item How to combine multimodal information?
      \end{itemize}
    \item Two tasks:
      \begin{itemize}
        \item Specific object recognition.
        \item Generic category recognition.
      \end{itemize}
  \end{itemize}
\end{frame}

\section{Feature Extraction}
\begin{frame}
  \frametitle{Bag-of-Words Model with SIFT Descriptors}

  \begin{itemize}
    \item SIFT Descriptor~\cite{lowe_object_1999}. 
      \begin{itemize}
        \item Local feature descriptor invariant to several image transforms.
        \item Standard for object recognition in robotics.

          ~
        \item Does not describe the whole image/object.
        \item Does not encode general features of a category.
      \end{itemize}
      ~
  \end{itemize}
\end{frame}

\begin{frame}
  \frametitle{Bag-of-Words Model with SIFT Descriptors}

  \begin{itemize}
    \item Bag-of-words model~\cite{csurka_visual_2004}. 
      \begin{itemize}
        \item A document : a collection of words.
        \item Count occurrences and ignore order. 

          ~
        \item An image : a collection of local features.
        \item Vector quantization with codebook.
      \end{itemize}
      ~
  \end{itemize}
\end{frame}

\begin{frame}
  \frametitle{Mel-frequency Cepstrum Coefficients}

  \begin{itemize}
    \item DCT of log power spectrum at mel-scale.
    \item MFCCs are common features for audio processing, such as speech recognition and speaker recognition.
  \end{itemize}
\end{frame}

\section{Hidden Markov Models}
\begin{frame}
  \frametitle{Hidden Markov Models}

  \begin{itemize}
    \item HMM describes a distribution of a stochastic process~\cite{rabiner_fundamentals_1993}.
      \begin{itemize}
        \item Applications in speech recognition and gesture recognition. 
        \item Images and sound change over time during interaction.
      \end{itemize}
      ~

    \item HMM assumes there is a hidden state variable associated with the observation at each time.
      \begin{itemize}
        \item Observation sequence: \[ \mathbf{x} = x_1 x_2 \dots x_T . \]
        \item State sequence: \[ \mathbf{q} = q_1 q_2 \dots q_T . \]
      \end{itemize}
  \end{itemize}
\end{frame}

\begin{frame}
  \frametitle{Two assumptions}

  \centering
  \includegraphics[width=.5\textwidth]{hmm.tikz}

  \begin{itemize}
    \item The present observation (at a certain time $t$) depends only upon the present hidden state:
      \[
        P(x_t|q_1, \dots, q_t, x_1, \dots, x_{t-1},x_{t+1},\dots,x_T) = P(x_t|q_t), \quad t \in \mathbb{N} 
      \]
    \item (Markov property) The future hidden state depends only upon the present state:
      \[
        P(q_{t+1}|q_1, \dots, q_t, x_1, \dots, x_t) = P(q_{t+1}|q_t),\quad t \in \mathbb{N}
      \]
  \end{itemize}
\end{frame}

\begin{frame}
  \frametitle{Probability Evaluation}

  \begin{itemize}
    \item Bla~\cite{rabiner_fundamentals_1993}.
      \begin{itemize}
        \item bla; 
      \end{itemize}
      ~

    \item bla.
      \begin{itemize}
        \item bla.
      \end{itemize}
  \end{itemize}
\end{frame}

\begin{frame}
  \frametitle{Parameter Estimation}

  \begin{itemize}
    \item Bla~\cite{rabiner_fundamentals_1993}.
      \begin{itemize}
        \item bla; 
      \end{itemize}
      ~

    \item bla.
      \begin{itemize}
        \item bla.
      \end{itemize}
  \end{itemize}
\end{frame}

\section{Bimodal Object Recognition}
\begin{frame}
  \frametitle{Modality Fusion Methods}

  \begin{itemize}
    \item Bla~\cite{rabiner_fundamentals_1993}.
      \begin{itemize}
        \item bla; 
      \end{itemize}
      ~

    \item bla.
      \begin{itemize}
        \item bla.
      \end{itemize}
  \end{itemize}
\end{frame}

\begin{frame}
  \frametitle{Feature Fusion}

  \begin{itemize}
    \item Bla~\cite{rabiner_fundamentals_1993}.
      \begin{itemize}
        \item bla; 
      \end{itemize}
      ~

    \item bla.
      \begin{itemize}
        \item bla.
      \end{itemize}
  \end{itemize}
\end{frame}

\begin{frame}
  \frametitle{Decision Fusion}

  \begin{itemize}
    \item Bla~\cite{rabiner_fundamentals_1993}.
      \begin{itemize}
        \item bla; 
      \end{itemize}
      ~

    \item bla.
      \begin{itemize}
        \item bla.
      \end{itemize}
  \end{itemize}
\end{frame}

\section{Experiment}
\begin{frame}
  \frametitle{Experiment Setup}

  \begin{itemize}
    \item Bla~\cite{rabiner_fundamentals_1993}.
      \begin{itemize}
        \item bla; 
      \end{itemize}
      ~

    \item bla.
      \begin{itemize}
        \item bla.
      \end{itemize}
  \end{itemize}
\end{frame}

\begin{frame}
  \frametitle{Experiment Result}

  \begin{itemize}
    \item Bla~\cite{rabiner_fundamentals_1993}.
      \begin{itemize}
        \item bla; 
      \end{itemize}
      ~

    \item bla.
      \begin{itemize}
        \item bla.
      \end{itemize}
  \end{itemize}
\end{frame}

\section{Conclusion}
\begin{frame}
  \frametitle{Conclusion}

  \begin{itemize}
    \item Bla~\cite{rabiner_fundamentals_1993}.
      \begin{itemize}
        \item bla; 
      \end{itemize}
      ~

    \item bla.
      \begin{itemize}
        \item bla.
      \end{itemize}
  \end{itemize}
\end{frame}

\begin{frame}
  \frametitle{Future Work}

  \begin{itemize}
    \item Bla~\cite{rabiner_fundamentals_1993}.
      \begin{itemize}
        \item bla; 
      \end{itemize}
      ~

    \item bla.
      \begin{itemize}
        \item bla.
      \end{itemize}
  \end{itemize}
\end{frame}

\frame[c]{
  \frametitle{The End}
  \begin{center}
    Thank you for your attention.\\[1ex]
    Any question?\\[5ex]
  \end{center}
}

\appendix
\section*{References}
\begin{frame}[allowframebreaks]
  \frametitle{References}
  {\scriptsize
    \bibliography{thesis}
    \bibliographystyle{unsrt}  
  }
\end{frame}

\end{document}

